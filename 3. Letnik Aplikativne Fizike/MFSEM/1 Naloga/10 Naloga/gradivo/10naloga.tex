\documentclass[slovene,11pt,a4paper]{article}
\usepackage[margin=2cm,bottom=3cm,foot=1.5cm]{geometry}
\setlength{\parindent}{0pt}
\setlength{\parskip}{0.5ex}

\usepackage[pdftex]{graphicx}
\DeclareGraphicsExtensions{.pdf,.png}


\usepackage{amsmath}
\usepackage{amsfonts}
\usepackage{mathrsfs}
\usepackage[usenames]{color}
\usepackage[slovene]{babel}
\usepackage[utf8]{inputenc}
\usepackage{siunitx}
\include{template}



\title{
\sc\large Matematično-fizikalni seminar \thisyear\\
\bigskip
\bf 10. naloga: Fourierova analiza}

\author{}
\date{}

\begin{document}
\maketitle
\vspace{-1cm}


Pri numeričnem izračunavanju Fourierove transformacije
\begin{equation}
H(f) = \int_{-\infty}^\infty
h(t)\,{\rm exp}(2 \pi i f t)\,dt
\end{equation}
\begin{equation}
h(t) = \int_{-\infty}^\infty
H(f)\,{\rm exp}(-2 \pi i f t)\,df
\end{equation}
je funkcija $h(t)$ običajno predstavljena s tablico diskretnih
vrednosti
\begin{equation}
h_k = h(t_k),\quad t_k = k \Delta, \quad k=0,1,2,\dots N-1.
\end{equation}
Pravimo, da smo funkcijo vzorčili z vzorčno gostoto $1/\Delta$.
Za tako definiran vzorec obstaja naravna meja frekvenčnega spektra,
ki se imenuje {\sl Nyquistova frekvenca}, $f_c =1/(2\Delta)$:
harmonični val s to frekvenco ima v naši vzorčni gostoti ravno
dva vzorca v periodi.
če ima funkcija $h(t)$ frekvenčni spekter omejen na interval
$[-f_c ,f_c ]$, potem ji z vzorčenjem nismo odvzeli nič informacije:
kadar pa se spekter razteza izven intervala, pride do {\sl potujitve\/}
({\sl aliasing\/}), ko se zunanji del spektra preslika v interval.

Frekvenčni spekter vzorčene funkcije (3) spet računamo samo
v $N$ točkah, če hočemo, da se ohrani količina informacije.
Vpeljemo vsoto
\begin{equation}
H_n = \sum_{k=0}^{N-1}
h_k\, {\rm exp}(2 \pi i k n / N),
\qquad n=-N/2,\dots ,N/2,
\end{equation}
ki jo imenujemo diskretna Fourierova transformacija, in je povezana
s funkcijo v (1) takole:
\[
H(n/(N\Delta)) \approx \Delta\cdot H_n .
\]
Zaradi potujitve, po kateri je $H_{-n} = H_{N-n}$, lahko mirno pustimo
indeks $n$ v enačbi (4) teči tudi od 0 do $N$. Spodnja polovica
tako definiranega spektra ($1 \le n \le \tfrac{N}{2}-1$) ustreza pozitivnim
frekvencam $0 < f < f_c$, gornja polovica ($\tfrac{N}{2}+1 \le N-1$)
pa negativnim, $-f_c < f < 0$.  Posebna vrednost pri $n=0$
ustreza frekvenci nič (``istosmerna komponenta''), vrednost
pri $n=N/2$ pa ustreza tako $f_c$ kot $-f_c$.

\par\medskip
Količine $h$ in $H$ so v splošnem kompleksne, simetrija
v enih povzroči tudi simetrijo v drugih.  Posebej zanimivi
so trije primeri:\par\medskip
\begin{tabular}{@{\hspace{1cm}}l@{\hspace{1cm}}l@{\hspace{1cm}}l@{\hspace{1cm}}l}
če je& $h_k$ realna & tedaj je & $H_{N-n} = H_n^\ast$ \\
      & $h_k$ realna in soda & & $H_n$ realna in soda \\
      & $h_k$ realna in liha & & $H_n$ imaginarna in liha
\end{tabular}
\par\medskip
(ostalih ni težko izpeljati).
V tesni zvezi s frekvenčnim spektrom je tudi moč.
{\sl Celotna moč\/} nekega signala je neodvisna od
reprezentacije, Parsevalova enačba pove
\begin{equation*}
\sum_{k=0}^{N-1} | h_k |^2 = {1\over N}\sum_{n=0}^{N-1} | H_n |^2
\end{equation*}
(lahko preveriš).  Pogosto pa nas bolj zanima, koliko moči
je vsebovane v frekvenčni komponenti med $f$ in $f+\dd f$, zato
definiramo enostransko spektralno gostoto moči ({\sl one-sided
power spectral density\/}, PSD)
\begin{equation*}
P_n = | H_n |^2 + | H_{N-n} |^2 \>.
\end{equation*}
Pozor: s takšno definicijo v isti koš mečemo negativne
in pozitivne frekvence, vendar sta pri realnih signalih $h_k$
prispevka enaka, tako da je $P_n = 2\,| H_n |^2$.
\par\medskip
Z diskretno obratno transformacijo lahko rekonstruiramo $h_k$ iz $H_n$
\begin{equation}
h_k = {1\over N} \sum_{n=0}^{N-1} H_n\, {\rm exp}(-2 \pi i k n / N).
\end{equation}
Količine $h$ in $H$ so v splošnem kompleksne, simetrija v enih
povzroči tudi simetrijo v drugih. Za realne $h$ so $H$ paroma
konjugirani, $H(-f) = H(f)^*$.

\medskip

{\it Naloga\/}:
\begin{itemize}

\item Izračunaj Fourierov obrat nekaj enostavnih vzorcev, npr. raznih mešanic izbranih frekvenc
(kombinacija sinusnih in kosinusnih nihanj). Primerjaj rezultate, ko je vzorec v intervalu periodičen
(izbrane frekvence so mnogokratniki osnovne frekvence), z rezultati, ko vzorec ni periodičen.
Opazuj pojav potujitve na vzorcu, ki vsebuje frekvence nad Nyquistovo frekvenco. Napravi še obratno
transformacijo (5) in preveri natančnost metode (kako dobro se rekonstruirani vzorec ujema z originalnim).
Preveri še odvisnost rezultata od dolžine vzorca in števila točk v Fourierovi analizi.
\item Dodatno: iz datoteke `waveform.txt' (najdeš jo na Spletni učilnici predmeta) izračunaj
Fourierov obrat zvočnega signala, ki je zabeležen v tej datoteki. Signal je bil posnet s frekvenco
44100 Hz, zato je Nyquistova frekvenca 22050 Hz. Signal je bil dodan tudi beli šum. Ali lahko ugotoviš,
katere frekvence so prisotne v signalu?

% Mihovilovic
% \item Signalu v datoteki {\tt PodatkiFourier.dat} \footnote{Podatke najdeš v Spletni učilnici predmeta} določi
% frekvenčni spekter.  Kako se spremeni spekter, če analiziramo krajše intervale (64, 128, $\ldots$ točk)?
% Kako se spremeni spekter, če v analizi upoštevamo vsako 2, 4, $\ldots$  točko?

%\item Izračunaj Fourierov obrat Gaussove porazdelitve in še nekaj enostavnih vzorcev,
%npr. raznih mešanic izbranih frekvenc. Primerjaj rezultate, ko
%je vzorec v intervalu periodičen (izbrane frekvence so mnogokratniki
%osnovne frekvence), z rezultati, ko vzorec ni periodičen.
%Opazuj pojav potujitve na vzorcu, ki vsebuje frekvence nad Nyquistovo
%frekvenco. Napravi še obratno transformacijo (5) in preveri
%natančnost metode.

\end{itemize}

\end{document}